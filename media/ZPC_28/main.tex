\documentclass[12pt]{exam}
\usepackage[utf8]{inputenc}
\usepackage[a4paper,top=20mm]{geometry}
\usepackage[fleqn]{mathtools}
\usepackage{xcolor}
\usepackage{hyperref}
\usepackage{setspace}
\usepackage{amsfonts}
\usepackage{graphics}
\usepackage{tikz}
\graphicspath{ {./images/} }
\usepackage{graphicx}
\usepackage{amssymb}
\usepackage{amsmath}
\usepackage{parskip}
\newcommand\ddfrac[2]{\frac{\displaystyle #1}{\displaystyle #2}}
\newcommand*\circled[1]{\tikz[baseline=(char.base)]{
            \node[shape=circle,draw,inner sep=2pt] (char) {#1};}}
\DeclareMathOperator{\Deg}{deg}
\title{ZPC-28}
\author{Bikramjyot, Farhan, Ishaan, Rachit}
\date{September 13, 2023}
\doublespacing
\begin{document}

\maketitle
\rule{\textwidth}{1pt}
\printanswers

\title{\textbf{\underline{\fontsize{18}{12}\selectfont Instructions}}}\\
Please read the following instructions carefully before proceeding further:\\
\begin{enumerate}
\item The test is of 2 hours. It will end \textit{sharp} at \textbf{8:00 pm}.
\item Relevant reading material for all the questions has been provided in the document itself.
\item If you are stuck on a question that you cannot figure out, move on. Nothing good ever comes out of being hung up on something.
\item Please feel free to contact the invigilators in case of any queries.
\item We will provide hints to questions in the last hour, depending on the participation and responses.
\item All the best. GL HF :)\\
\end{enumerate}
\bigskip
\maketitle
\newpage
\title{\begin{center}\textbf{\underline{\fontsize{16}{12}\selectfont GENERATING FUNCTIONS}}\end{center}}
\section{Generating Functions}

A (one-variable) generating function for a sequence $a_{0}, a_{1}, a_{2}, \ldots$ is the formal power series $a_{0}+a_{1} x+a_{2} x^{2}+\cdots$. Generating functions are useful for solving recurrences, counting certain combinatorial objects, and even just finding a nice formula for the generating function itself.

The classic example of a generating function identity is

$$
\frac{1}{1-x}=1+x+x^{2}+x^{3}+\cdots
$$

Ordinarily, we think of this as the geometric series formula, which converges for $|x|<1$, but in the world of generating functions we are more concerned with the coefficients of the series than with the values of $x$.

Definition. A formal power series over the variable $x$ is simply an expression of the form $c_{0}+c_{1} x+c_{2} x^{2}+\cdots=\sum_{i=0}^{\infty} c_{i} x^{i}$ where each $c_{i}$ is a complex number. ${ }^{1}$

We define addition and multiplication of formal power series as follows:

$$
\begin{gathered}
\sum_{i=0}^{\infty} a_{i} x^{i}+\sum_{i=0}^{\infty} b_{i} x^{i}=\sum_{i=0}^{\infty}\left(a_{i}+b_{i}\right) x^{i} \\
\left(\sum_{i=0}^{\infty} a_{i} x^{i}\right) \cdot\left(\sum_{i=0}^{\infty} a_{i} x^{i}\right)=\sum_{n=0}^{\infty}\left(\sum_{i=0}^{n} a_{i} b_{n-i}\right) x^{n}
\end{gathered}
$$

Exercise. Use the definition of formal power series multiplication to prove that the identity

$$
\frac{1}{1-x}=1+x+x^{2}+x^{3}+\cdots
$$

holds when thought of as formal power series over $x$.

We also define the derivative of a formal power series by

$$
\frac{d}{d x}\left(\sum_{n=0}^{\infty} a_{n} x^{n}\right)=\sum_{i=1}^{\infty} n a_{n} x^{n-1} .
$$

The derivative behaves exactly like ordinary derivatives from calculus.

${ }^{1}$ We will usually, however, encounter the case where each $c_{i}$ is an integer. Exercise. Let $F(x)$ and $G(x)$ be generating functions (formal power series over $x$ ). Use the rules for formal manipulation of power series to prove that the derivative satisfies:

- $\frac{d}{d x}(F(x)+G(x))=\frac{d}{d x} F(x)+\frac{d}{d x} G(x)$

- $\frac{d}{d x}(F(x) G(x))=G(x) \cdot \frac{d}{d x} F(x)+F(x) \cdot \frac{d}{d x} G(x)$

- $\frac{d}{d x}(F(x) / G(x))=\left(G(x) \frac{d}{d x} F(x)-F(x) \frac{d}{d x} G(x)\right) / G(x)^{2}$.

Taking derivatives enables us to find new generating functions from old ones. For instance, taking the derivative of both sides of the geometric series formula yields

$$
\frac{1}{(1-x)^{2}}=1+2 x+3 x^{2}+4 x^{3}+5 x^{4}+\cdots .
$$

This, in fact, holds for $|x|<1$, which is somewhat harder to show.

\section{Using generating functions to solve recurrences}

Suppose we wish to find an explicit formula for the $n$th Fibonnacci number $F_{n}$, where $F_{0}=0, F_{1}=1$, and $F_{n}=F_{n-1}+F_{n-2}$ for all $n \geq 2$. Consider the generating function

$$
G(x)=\sum_{n=0}^{\infty} F_{n} x^{n}
$$

We can manipulate this to take advantage of the recursion: we have

$$
G(x)-x G(x)-x^{2} G(x)=F_{0}+F_{1} x-F_{0} x+\sum_{n=2}^{\infty}\left(F_{n}-F_{n-1}-F_{n-2}\right) x^{n}=x .
$$

Thus $G(x)=x /\left(1-x-x^{2}\right)$. Using partial fractions and expanding each term as a geometric series, we find that

$$
G(x)=\sum_{n=0}^{\infty} \frac{1}{\sqrt{5}}\left(\left(\frac{1+\sqrt{5}}{2}\right)^{n}-\left(\frac{1-\sqrt{5}}{2}\right)^{n}\right) x^{n},
$$

and so $F_{n}=\frac{1}{\sqrt{5}}\left(\left(\frac{1+\sqrt{5}}{2}\right)^{n}-\left(\frac{1-\sqrt{5}}{2}\right)^{n}\right)$.

In general, the generating function for any linear recurrence of the form

$$
A_{n}=c_{1} A_{n-1}+c_{2} A_{n-2}+\cdots+c_{k} A_{n-k}
$$

can be written as a rational function of $x$, obtained by multiplying it by the characteristic polynomial

$$
1-c_{1} x-c_{2} x^{2}-\cdots-c_{k} x^{k}
$$

and using the initial conditions to solve for the generating function. We can then use partial fraction decomposition and the geometric series formula to find an explicit formula for the $n$th coefficient. 

\section{Exponential generating functions}

The exponential generating function for the sequence $\left\{a_{i}\right\}_{i=0}^{\infty}$ is the series $\sum_{n=0}^{\infty} \frac{a_{n}}{n !} x^{n}$. Their product behaves somewhat differently from that of ordinary generating functions:

$$
\left(\sum_{n=0}^{\infty} \frac{a_{n}}{n !} x^{n}\right) \cdot\left(\sum_{n=0}^{\infty} \frac{b_{n}}{n !} x^{n}\right)=\sum_{n=0}^{\infty} \frac{1}{n !}\left(\sum_{k=0}^{n}\left(\begin{array}{l}
n \\
k
\end{array}\right) a_{k} b_{n-k}\right) x^{n}
$$

We define $e^{x}, \sin (x)$, and $\cos (x)$ to be the exponential generating functions shown below. Interpreting these as formal power series, they satisfy all the trigonometric identities one would expect.

- $e^{x}=\sum_{n=0}^{\infty} \frac{1}{n !} x^{n}$

- $\sin (x)=\sum_{n=0}^{\infty}(-1)^{n} \frac{1}{(2 n+1) !} x^{2 n+1}$

- $\cos (x)=\sum_{n=0}^{\infty}(-1)^{n} \frac{1}{(2 n) !} x^{2 n}$

Exercise. Using the definition of $e^{x}$ as a formal power series, show that $e^{x} e^{y}=e^{x+y}$.

\section{Problems}
\begin{questions}
\question Find the generating function for each of the following sequences, and use it to find an explicit formula for $a_{n}$ :
\begin{parts}
\part $a_{0}=1, a_{1}=5, a_{n+2}=4 a_{n+1}-3 a_{n}$
\begin{solution} $$ a_n = 2\times3^n +1$$
\end{solution}
\part $a_{0}=1, a_{1}=6, a_{n+2}=4 a_{n+1}-4 a_{n}$
\begin{solution} $$ a_n = 2^n(2n+1)$$
\end{solution}
\part $a_{0}=0, a_{1}=5, a_{2}=47, a_{n+3}=31 a_{n+1}+30 a_{n}$
\begin{solution} $$a_n = \frac{1}{2}((-5)^n-3\times(-1)^n)+6^n$$
\end{solution}
\part $a_{0}=0, a_{1}=1, a_{n+2}=a_{n+1}+2 a_{n}+1$
\begin{solution} $$a_n = \frac{1}{6}((-1)^{n+1}+2^{n+2}-2)$$
\end{solution}
\part $a_{0}=a_{1}=1, a_{n+2}=a_{n+1}+6 a_{n}+n$
\begin{solution} $$a_n = \frac{1}{180} (-6 (2 (-1)^{2n} + 3) n + 17 (-1)^n 2^{n + 2} + 13\times 3^{n + 2} - 5)$$
\end{solution}
\end{parts}

\question \begin{parts}
    
\part Let $D_{n}$ be the number of derangements of $n$, that is, the number of permutations $\phi$ of $\{1,2, \ldots, n\}$ such that $\phi(i) \neq i$ for any $1 \leq i \leq n$. Find a closed form expression for the exponential generating function of $D_{n}$, and use it to find a formula for $D_{n}$ (the formula may include a finite sum.)
\begin{solution}
Prove that $D_{n}=nD_{n-1}+(-1)^{n}$ \newline
$\implies D_n \frac{x^n}{n!} = nD_{n-1} \frac{x^n}{n!} + (-1)^n \frac{x^n}{n!}$ \newline
$\displaystyle \implies \sum_{n\geq 1} D_n \frac{x^n}{n!} = \sum_{n\geq 1} nD_{n-1} \frac{x^n}{n!} + \sum_{n\geq 1} (-1)^n \frac{x^n}{n!}$ \newline
$\displaystyle \implies G_D(x)-1 = x \sum_{n\geq 1} D_{n-1} \frac{x^{n-1}}{(n-1)!} + e^x-1 $ \newline
$\implies G_D(x) = xG_D(x) + e^x-1$ \newline
$\implies G_D(x)=\frac{e^x}{1-x}$
\end{solution}

\part Let $P_{n}$ be the number of ways a $2 \times 2 \times n$ pillar can be built out of $2 \times 1 \times 1$ bricks . Find a closed form expression for the generating function $\displaystyle \sum_{n=0}^{\infty} P_{n} x^{n}$.
\begin{solution}
The key here is realizing that breaking or desired recurrence into smaller recurrences can help us simplify it ($a_n$ computes the number of ways to tile the entire pillar and $i_n$ computes the number of ways to tile the entire pillar except two adjacent positions at the top layer)

$a_0=1$, $a_1=2$, $a_2=9$, $a_n=2a_{n-1}+a_{n-2}+i_{n-1}$ $\forall$ $n \geq 2$ \newline
$i_0=0$, $i_1=0$, $i_2=4$, $i_n=i_{n-1}+4a_{n-2}$ $\forall$ $n \geq 2$

$a_n-a_{n-1}=2(a_{n-1}-a_{n-2})+(a_{n-2}-a_{n-3})+(i_n-i_{n-1})$ $\forall$ $n \geq 3$ \newline
$a_n=3a_{n-1}+3a_{n-2}-a_{n-3}$ $\forall$ $n \geq 3$

$\displaystyle \sum_{n=0}^{\infty} P_{n} x^{n} = \sum_{n=0}^{\infty} (\frac{1-n}{1-3n-3n^2+n^3}) x^{n}$
\end{solution}

\end{parts}

\question Prove that:
$$
\sum_{\substack{i+j=n \\
i, j \geq 0}}\left(\begin{array}{c}
2 i \\
i
\end{array}\right)\left(\begin{array}{c}
2 j \\
j
\end{array}\right)=4^{n}
$$

\begin{solution}
The key here is realizing that this summation is the coefficient of $x^n$ in the multiplication of two identical generating functions for sequences, where each term $a_n={2n \choose n}$, so the generating function of the LHS is the square of the generating function for $a_n$.

Assuming there exists $a \in \mathbb{R}$ and $n \in \mathbb{R}$ such that: \newline $\displaystyle (1+ax)^n=\sum_{n=0}^{\infty} {2n \choose n} x^n=\sum_{n=0}^{\infty} \frac{2n!}{n!n!} x^n$ \newline
Differentiating we get: $\displaystyle na(1+ax)^{n-1}=\sum_{n=1}^{\infty} \frac{2n!n}{n!n!} x^{n-1}$ \newline
Differentiating again: $\displaystyle n(n-1)a^2(1+ax)^{n-2}=\sum_{n=2}^{\infty} \frac{2n!n(n-1)}{n!n!} x^{n-2}$ \newline
$\implies$ $na=2$ and $n(n-1)a^2=12$ at $x=0$ \newline
$\implies$ $a=-4$ and $n=-1/2$ \newline
The generating function for $a_n={2n \choose n}$ is $f(x)=\frac{1}{\sqrt{1-4x}}$ and the generating function that we require is $f^{2}(x)=\frac{1}{1-4x}$

But $\frac{1}{1-4x}=1+4x+(4x)^2+(4x)^3+\dots$, based on which we can show that the coefficient of $x^n$ is $4^n$
\end{solution}

\question Find the number of subsets of $\{1,2, \ldots, 2000\}$, the sum of whose elements is divisible by 5 .
\begin{solution}
\href{https://math.stackexchange.com/questions/1618420/find-the-number-of-all-subsets-of-1-2-ldots-2015-with-n-elements-such}{StackExchange Answer}
\end{solution}

\question Let $n$ be a positive integer. Find the number of polynomials $P(x)$ with coefficients in $\{0,1,2,3\}$ such that $P(2)=n$.
\begin{solution}
\href{https://math.stackexchange.com/questions/3314422/find-the-number-of-polynomials-px-with-coefficients-0-1-2-3-such-that-p2}{StackExchange Answer}
\end{solution}

\question Let $p$ be an odd prime. Find the number of subsets $A$ of $\{1,2, \ldots, 2 p\}$ such that:
\begin{itemize}

\item $A$ has exactly $p$ elements, and

\item the sum of all the elements of $A$ is divisible by $p$.
\end{itemize}

\begin{solution}
\href{https://math.stackexchange.com/questions/314788/let-p-be-an-odd-prime-number-how-many-p-element-subsets-of-1-2-3-4-ldo}{StackExchange Answer}
\end{solution}
\end{questions}

$$\square \square \square \square \square \square \square \square \square \square \square \square \square \square \square $$
\end{document}

%https://www.dropbox.com/sh/w9mfy9qtjs68xzc/AAAOrtU5GcEmE01sbNoSDqzVa/Combinatorics/Generating%20Functions%20-%20Maria%20Monks%20-%20MOP%202010.pdf?dl=0